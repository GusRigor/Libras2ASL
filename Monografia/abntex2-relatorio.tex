%% abtex2-modelo-relatorio-tecnico.tex, v<VERSION> laurocesar
%% Copyright 2012-<COPYRIGHT_YEAR> by abnTeX2 group at http://www.abntex.net.br/ 
%%
%% This work may be distributed and/or modified under the
%% conditions of the LaTeX Project Public License, either version 1.3
%% of this license or (at your option) any later version.
%% The latest version of this license is in
%%   http://www.latex-project.org/lppl.txt
%% and version 1.3 or later is part of all distributions of LaTeX
%% version 2005/12/01 or later.
%%
%% This work has the LPPL maintenance status `maintained'.
%% 
%% The Current Maintainer of this work is the abnTeX2 team, led
%% by Lauro César Araujo. Further information are available on 
%% http://www.abntex.net.br/
%%
%% This work consists of the files abntex2-modelo-relatorio-tecnico.tex,
%% abntex2-modelo-include-comandos and abntex2-modelo-references.bib
%%

% ------------------------------------------------------------------------
% ------------------------------------------------------------------------
% abnTeX2: Modelo de Relatório Técnico/Acadêmico em conformidade com 
% ABNT NBR 10719:2015 Informação e documentação - Relatório técnico e/ou
% científico - Apresentação
% ------------------------------------------------------------------------ 
% ------------------------------------------------------------------------

\documentclass[
	% -- opções da classe memoir --
	12pt,				% tamanho da fonte
	openany,			% caps. começam em pág. ímpar, insere pág. vazia caso preciso openright ou openany
	oneside,			% para impressão em recto e verso. twoside ou oneside 
	a4paper,			% tamanho do papel. 
	% -- opções da classe abntex2 --
	%chapter=TITLE,		% títulos de capítulos convertidos em letras maiúsculas
	%section=TITLE,		% títulos de seções convertidos em letras maiúsculas
	%subsection=TITLE,	% títulos de subseções convertidos em letras maiúsculas
	%subsubsection=TITLE,% títulos de subsubseções convertidos em letras maiúsculas
	% -- opções do pacote babel --
	english,			% idioma adicional para hifenização
	french,				% idioma adicional para hifenização
	spanish,			% idioma adicional para hifenização
	brazil,				% o último idioma é o principal do documento
	sumario=tradicional]{abntex2}


% ---
% PACOTES
% ---

% ---
% Pacotes fundamentais 
% ---
\usepackage{lmodern}			% Usa a fonte Latin Modern
\usepackage[T1]{fontenc}		% Selecao de codigos de fonte.
\usepackage[utf8]{inputenc}		% Codificacao do documento (conversão automática dos acentos)
\usepackage{indentfirst}		% Indenta o primeiro parágrafo de cada seção.
\usepackage{color}				% Controle das cores
\usepackage{graphicx}			% Inclusão de gráficos
\usepackage{microtype} 			% para melhorias de justificação
\usepackage{float}              % fixar gráfico
% ---

% ---
% Pacotes adicionais, usados no anexo do modelo de folha de identificação
% ---
\usepackage{multicol}
\usepackage{multirow}
% ---

% ---
% E cursivo
% ---
\usepackage{amsmath}
\DeclareFontFamily{U}{calligra}{}
\DeclareFontShape{U}{calligra}{m}{n}{<->callig15}{}
\newcommand{\calE}{{\!\!\text{\usefont{U}{calligra}{m}{n}E}\,\,}}
\usepackage{subcaption}
	
% ---
% Pacotes adicionais, usados apenas no âmbito do Modelo Canônico do abnteX2
% ---
\usepackage{lipsum}				% para geração de dummy text
% ---

% ---
% Pacotes de citações
% ---
\usepackage[brazilian,hyperpageref]{backref}	 % Paginas com as citações na bibl
\usepackage[alf]{abntex2cite}	% Citações padrão ABNT

% --- 
% CONFIGURAÇÕES DE PACOTES
% --- 

% ---
% Configurações do pacote backref
% Usado sem a opção hyperpageref de backref
\renewcommand{\backrefpagesname}{Citado na(s) página(s):~}
% Texto padrão antes do número das páginas
\renewcommand{\backref}{}
% Define os textos da citação
\renewcommand*{\backrefalt}[4]{
	\ifcase #1 %
		Nenhuma citação no texto.%
	\or
		Citado na página #2.%
	\else
		Citado #1 vezes nas páginas #2.%
	\fi}%
% ---

% ---
% Informações de dados para CAPA e FOLHA DE ROSTO
% ---
\instituicao{%
  Universidade Federal de São Carlos -- UFSCar
  \par
  Centro de Ciências Exatas e de Tecnologia
  \par
  Departamento de Física
  \par
  97527 - Estágio Curricular 2}
\titulo{Relatório de Estágio}
\autor{Felipe de Souza Custódio}
\local{São Paulo, Brasil}
\data{\today}
\tipotrabalho{Relatório técnico}
% O preambulo deve conter o tipo do trabalho, o objetivo, 
% o nome da instituição e a área de concentração 
\preambulo{Estágio no Centro Nacional de Pesquisa em Energia e Materiais na área de simulação do Grupo de Óptica do Laboratório Nacional de Luz Síncrona. \newline Orientador: Bernd Christian Meyer \newline Professor: Fernando Manuel Araújo Moreira}
% ---

% ---
% Configurações de aparência do PDF final


% alterando o aspecto da cor azul
\definecolor{blue}{RGB}{41,5,195}

% informações do PDF
\makeatletter
\hypersetup{
     	%pagebackref=true,
		pdftitle={\@title}, 
		pdfauthor={\@author},
    	pdfsubject={\imprimirpreambulo},
	    pdfcreator={LaTeX with abnTeX2},
		pdfkeywords={abnt}{latex}{abntex}{abntex2}{relatório técnico}, 
		colorlinks=true,       		% false: boxed links; true: colored links
    	linkcolor=blue,          	% color of internal links
    	citecolor=blue,        		% color of links to bibliography
    	filecolor=magenta,      	% color of file links
		urlcolor=blue,
		bookmarksdepth=4}
\makeatother
% --- 

% --- 
% Espaçamentos entre linhas e parágrafos 
% --- 

% O tamanho do parágrafo é dado por:
\setlength{\parindent}{1.5cm}

% Controle do espaçamento entre um parágrafo e outro:
\setlength{\parskip}{0.2cm}  % tente também \onelineskip

% ---
% compila o indice
% ---
\makeindex
% ---

% ----
% Início do documento
% ----
\begin{document}

% Seleciona o idioma do documento (conforme pacotes do babel)
%\selectlanguage{english}
\selectlanguage{brazil}

% Retira espaço extra obsoleto entre as frases.
\frenchspacing 

% ----------------------------------------------------------
% ELEMENTOS PRÉ-TEXTUAIS
% ----------------------------------------------------------
% \pretextual

% ---
% Capa
% ---
\imprimircapa
% ---

% ---
% Folha de rosto
% (o * indica que haverá a ficha bibliográfica)
% ---
\imprimirfolhaderosto*
% ---


%% ---
% Agradecimentos
% ---
%\begin{agradecimentos}
%Agradecimentos
%
%\end{agradecimentos}
% ---

% ---
% RESUMO
% ---

% resumo na língua vernácula (obrigatório)
\setlength{\absparsep}{18pt} % ajusta o espaçamento dos parágrafos do resumo
\begin{resumo}
\lipsum[50]

\noindent
\textbf{Palavras-chaves}: Palavras. Chaves
\end{resumo}
% ---

% ---
% inserir lista de ilustrações
% ---
\pdfbookmark[0]{\listfigurename}{lof}
\listoffigures*
\cleardoublepage
% ---

% ---
% inserir lista de tabelas
% ---
\pdfbookmark[0]{\listtablename}{lot}
\listoftables*
\cleardoublepage
% ---

% ---
% inserir lista de abreviaturas e siglas
% ---
%\begin{siglas}
%	\item[CNPEM] Centro Nacional de Pesquisa em Energia e Materiais
%\end{siglas}
% ---

% ---
% inserir lista de símbolos
% ---
%\begin{simbolos}
%   \item[$ \beta $] Letra grega beta
% 	\item[$ \Gamma $] Letra grega Gama
% 	\item[$ \Lambda $] Lambda
% 	\item[$ \zeta $] Letra grega minúscula zeta
% 	\item[$ \in $] Pertence
%\end{simbolos}
% ---

% ---
% inserir o sumario
% ---
\pdfbookmark[0]{\contentsname}{toc}
\tableofcontents*
\cleardoublepage
% ---


% ----------------------------------------------------------
% ELEMENTOS TEXTUAIS
% ----------------------------------------------------------
\textual

% ----------------------------------------------------------
% Introdução (exemplo de capítulo sem numeração, mas presente no Sumário)
% ----------------------------------------------------------
\chapter*[Introdução]{Introdução}
\addcontentsline{toc}{chapter}{Introdução}
\lipsum[50]

% ----------------------------------------------------------
% PARTE - preparação da pesquisa
% ----------------------------------------------------------
\part{Preparação do relatório}

\chapter{Objetivos}
\lipsum[24]


\chapter{Materiais e Métodos}
\lipsum[24] \cite{Sigma}

\section{Fonte de Luz Síncrotron}
\lipsum[24]

\subsection{Fontes}
\lipsum[24]


% ----------------------------------------------------------
% Parte de resultados
% ----------------------------------------------------------
\part{Resultados}\label{Resultados}

\chapter{Resultados obtidos}
\lipsum[24]

\section{Fonte}
\lipsum[24]

\begin{table}[H]
\centering
\ABNTEXfontereduzida
\caption{Ondulador da Kyma}
\label{tab:Kyma_parameters}
\begin{tabular}{@{}lrl@{}}
\toprule
\multicolumn{3}{c}{Ondulador Kyma}     \\ 
\midrule
Período         & 58.0   & GeV \\
L               & 1.044  &     \\
N de Períodos   & 18     & mA  \\ 
K$_{MAX}$         & 5.2    &     \\
B$_{MAX}$         & 0.9585 &     \\
Polarização     & Horizontal  &     \\
Gap             & 15.8   &     \\ \bottomrule
\end{tabular}
\end{table}


\chapter{Discussão dos Resultados}
\lipsum[24]


% ---
% Finaliza a parte no bookmark do PDF
% para que se inicie o bookmark na raiz
% e adiciona espaço de parte no Sumário
% ---
\phantompart

% ---
% Conclusão
% ---
\chapter{Conclusão}
\textcolor{red}{Concluímos que \lipsum[30]}

% ----------------------------------------------------------
% Capitulo com exemplos de comandos inseridos de arquivo externo 
% ----------------------------------------------------------
%\part{ABNTEX}
%\include{abntex2-modelo}

% ----------------------------------------------------------
% ELEMENTOS PÓS-TEXTUAIS
% ----------------------------------------------------------
\postextual

% ----------------------------------------------------------
% Referências bibliográficas
% ----------------------------------------------------------
\bibliography{references}

% ----------------------------------------------------------
% Glossário
% ----------------------------------------------------------
%
% Consulte o manual da classe abntex2 para orientações sobre o glossário.
%
%\glossary

% ----------------------------------------------------------
% Apêndices
% ----------------------------------------------------------

% ---
% Inicia os apêndices
% ---
\begin{apendicesenv}

% Imprime uma página indicando o início dos apêndices
\partapendices

% ----------------------------------------------------------
\chapter{Quisque libero justo}

\lipsum[50]
% ----------------------------------------------------------

\end{apendicesenv}
% ---


% ----------------------------------------------------------
% Anexos
% ----------------------------------------------------------

% ---
% Inicia os anexos
% ---
%\begin{anexosenv}
% Imprime uma página indicando o início dos anexos
%\partanexos
% ---
%\chapter{Parâmetros da radiação síncrotron do SIRIUS}\label{Sirius_parametros}
%\end{anexosenv}
% ---

%---------------------------------------------------------------------
% INDICE REMISSIVO
%---------------------------------------------------------------------
%
%\phantompart
%
%\printindex

\end{document}
